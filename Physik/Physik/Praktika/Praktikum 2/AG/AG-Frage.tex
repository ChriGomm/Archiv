\documentclass[a4paper, 12pt]{minimal}



\usepackage[utf8]{inputenc}

\usepackage[ngerman]{babel}
\usepackage{amssymb}
\usepackage[T1]{fontenc}
\usepackage{mathtools}
\usepackage{amsmath}
\usepackage{ntheorem}
\usepackage{bbm}
\usepackage{dsfont}
\usepackage{color}
\usepackage{slashed}
\usepackage{hyperref}
\usepackage{graphicx} 
\usepackage{bm}
\usepackage{mathabx}
\usepackage{float}
\usepackage{mwe}
\usepackage{multirow}
\begin{document}
${ }$\newline
Hi Gina,\newline\newline
ich habe ein Problem bei der Herleitung der 4. Formel der Anleitung
\begin{equation*}\frac{\lambda}{\Delta\lambda}=\frac{B\,\epsilon}{\Delta\lambda}\end{equation*}
Das Prisma erzeugt doch einen Unterschied der beiden Wellenfronten unterschiedlicher Wellenlänge im Winkel zu einander. Dieser Winkel zwischen den Wellenfronten erzeugt beim Auftreffen auf den Messspalt einen Gangunterschied der sich nach Durchgang durch den Messspalt nicht mehr Ausgleichen lässt. Wenn also Licht der Wellenlänge $\lambda+\Delta\lambda$ an einer Stelle hinter dem Schirm ein Beugungsmaximum hat, dann muss das Licht der Wellenlänge der $\lambda$ an diese Stelle gerade das dem Beugungsmaximum gleicher Ordnung vorangehende Minimum haben. Das wäre zumindest die Bedingung für die minimale Möglichkeit die beiden Beugungsmaxima aufzulösen. Da der Unterschied im Weg zwischen einem Beugungsminimum und einem Beugungsmaximum am Einfachspalt gerade $\lambda/2$ ist, und die beiden betrachteten Wellenlängen von der Wellenlänge sehr nahe beieinander liegen, kann man sagen, dass der Wegunterschied, der durch die Ankunft am Spalt unter verschiedenen Winkel herrührt, $\lambda/2$ betragen muss. Dieser Wegunterschied ist 
\begin{equation*}\sin(\epsilon)\cdot{B}\end{equation*}
wobei $\epsilon$ der Winkel zischen den Wellenfronten der beiden Wellenlängen (die beiden orangenen in unserem Fall), der durch den Durchgang durch das Prisma entstanden ist, ist. Mit der Kleinwinkelnäherung führt dies auf
\begin{equation*}\frac{\lambda}{\Delta\lambda}=2\,\frac{B\cdot\epsilon}{\Delta\lambda}\end{equation*}
Nach dieser Herleitung müsste also ein Faktor 2 noch in der Formel sein, der in der Anleitung nicht da ist. Das ist ein Problem.







\end{document}